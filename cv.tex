%-----------------------------------------------------------------------------------------------------------------------------------------------%
%	The MIT License (MIT)
%
%	Copyright (c) 2021 Jitin Nair
%
%	Permission is hereby granted, free of charge, to any person obtaining a copy
%	of this software and associated documentation files (the "Software"), to deal
%	in the Software without restriction, including without limitation the rights
%	to use, copy, modify, merge, publish, distribute, sublicense, and/or sell
%	copies of the Software, and to permit persons to whom the Software is
%	furnished to do so, subject to the following conditions:
%	
%	THE SOFTWARE IS PROVIDED "AS IS", WITHOUT WARRANTY OF ANY KIND, EXPRESS OR
%	IMPLIED, INCLUDING BUT NOT LIMITED TO THE WARRANTIES OF MERCHANTABILITY,
%	FITNESS FOR A PARTICULAR PURPOSE AND NONINFRINGEMENT. IN NO EVENT SHALL THE
%	AUTHORS OR COPYRIGHT HOLDERS BE LIABLE FOR ANY CLAIM, DAMAGES OR OTHER
%	LIABILITY, WHETHER IN AN ACTION OF CONTRACT, TORT OR OTHERWISE, ARISING FROM,
%	OUT OF OR IN CONNECTION WITH THE SOFTWARE OR THE USE OR OTHER DEALINGS IN
%	THE SOFTWARE.
%	
%
%-----------------------------------------------------------------------------------------------------------------------------------------------%

%----------------------------------------------------------------------------------------
%	DOCUMENT DEFINITION
%----------------------------------------------------------------------------------------

% article class because we want to fully customize the page and not use a cv template
\documentclass[a4paper,12pt]{article}

%----------------------------------------------------------------------------------------
%	FONT
%----------------------------------------------------------------------------------------

% % fontspec allows you to use TTF/OTF fonts directly
% \usepackage{fontspec}
% \defaultfontfeatures{Ligatures=TeX}

% % modified for ShareLaTeX use
% \setmainfont[
% SmallCapsFont = Fontin-SmallCaps.otf,
% BoldFont = Fontin-Bold.otf,
% ItalicFont = Fontin-Italic.otf
% ]
% {Fontin.otf}

%----------------------------------------------------------------------------------------
%	PACKAGES
%----------------------------------------------------------------------------------------
\usepackage{url}
\usepackage{parskip} 	

%other packages for formatting
\RequirePackage{color}
\RequirePackage{graphicx}
\usepackage[usenames,dvipsnames]{xcolor}
\usepackage[scale=0.9]{geometry}

%tabularx environment
\usepackage{tabularx}

%for lists within experience section
\usepackage{enumitem}

% centered version of 'X' col. type
\newcolumntype{C}{>{\centering\arraybackslash}X} 

%to prevent spillover of tabular into next pages
\usepackage{supertabular}
\usepackage{tabularx}
\newlength{\fullcollw}
\setlength{\fullcollw}{0.47\textwidth}

%custom \section
\usepackage{titlesec}				
\usepackage{multicol}
\usepackage{multirow}

%CV Sections inspired by: 
%http://stefano.italians.nl/archives/26
\titleformat{\section}{\Large\scshape\raggedright}{}{0em}{}[\titlerule]
\titlespacing{\section}{0pt}{10pt}{10pt}

%for publications
\usepackage[style=authoryear,sorting=ynt, maxbibnames=2]{biblatex}

%Setup hyperref package, and colours for links
\usepackage[unicode, draft=false]{hyperref}
\definecolor{linkcolour}{rgb}{0,0.2,0.6}
\hypersetup{colorlinks,breaklinks,urlcolor=linkcolour,linkcolor=linkcolour}
\addbibresource{citations.bib}
\setlength\bibitemsep{1em}

%for social icons
\usepackage{fontawesome5}

%debug page outer frames
%\usepackage{showframe}
\newcommand{\signed}[1]{%
\unskip\nobreak\hfil\penalty50
   \hskip2em\hbox{}\nobreak\hfil#1
   \parfillskip=0pt \finalhyphendemerits=0 }
%----------------------------------------------------------------------------------------
%	BEGIN DOCUMENT
%----------------------------------------------------------------------------------------
\begin{document}

% non-numbered pages
\pagestyle{empty} 

%----------------------------------------------------------------------------------------
%	TITLE
%----------------------------------------------------------------------------------------

% \begin{tabularx}{\linewidth}{ @{}X X@{} }
% \huge{Your Name}\vspace{2pt} & \hfill \emoji{incoming-envelope} email@email.com \\
% \raisebox{-0.05\height}\faGithub\ username \ | \
% \raisebox{-0.00\height}\faLinkedin\ username \ | \ \raisebox{-0.05\height}\faGlobe \ mysite.com  & \hfill \emoji{calling} number
% \end{tabularx}

\begin{tabularx}{\linewidth}{@{} C @{}}
\Huge{zhuwenq} \\[7.5pt]
\href{https://github.com/leonezz}{\raisebox{-0.05\height}\faGithub\ Leonezz} \ $|$ \  
\href{https://zhuwenq-blog.netlify.app}{\raisebox{-0.05\height}\faGlobe \ Blog} \ $|$ \ 
\href{mailto:zhuwenqa@outlook.com}{\raisebox{-0.05\height}\faEnvelope \ zhuwenqa@outlook.com} \ $|$ \ 
\href{tel:+8618551713651}{\raisebox{-0.05\height}\faMobile \ +86 18551713651} \ $|$ \
\raisebox{-0.05\height}\faUserCircle \ Software Engineer \\
\end{tabularx}

%----------------------------------------------------------------------------------------
% EXPERIENCE SECTIONS
%----------------------------------------------------------------------------------------

%Interests/ Keywords/ Summary
\section{Summary}
I am currently a postgraduate major in computer science, my research interests include Natural Language Processing, Parameter Efficient Transfer Learning and Information Extraction. Before my postgraduate's study, I graduate from the optical major of physics faculty.

In addition to my academic study, I am a person of wide interests, my interests include modern C++, embedded systems, advanced deep learning technics, reading books, photography and making coffee.

%----------------------------------------------------------------------------------------
%	EDUCATION
%----------------------------------------------------------------------------------------
\section{Education}
\begin{tabularx}{\linewidth}{@{}l X@{}}	
2021 - present &Master Candidate at \textbf{NUAA} major in \textit{Computer Science and Technology}.
\signed{\normalsize{(GPA:)}}\\
2017 - 2021 &Bachelor's Degree at \textbf{NUAA} major in \textit{Optoelectronic Information Science and Engineering}.
\signed{\normalsize{(GPA: 3.2/5.0)}} \\
\end{tabularx}

%Experience
\section{Project Experience}

\begin{tabularx}{\linewidth}{ @{}X@{}  }
\textbf{A Measuring System of Metal Surface Flatness Based on Electromagnetic Induction Principle} as \textit{Creator and maintainer}.
\signed{Sept. 2018 - Nov. 2018}\\[3.75pt]
\begin{minipage}[t]{\linewidth}
    \begin{itemize}[nosep,after=\strut, leftmargin=1em, itemsep=3pt]
        \item[-] This project is supported by the NUAA Innovation and Entrepreneurship Program.
        \item[-] We designed a tiny PCB coil in this project as the electromagnetic sensor and power this sensor with a dedicated chip.
        \item[-] The dedicated chip collect the electromagnetic impedance value between the coil sensor and the surface of the object
        to be tested, and encode the value into digital data.
        \item[-] We collect the data that represents the impedance (i.e. the distance between sensor and the surface to be tested)
        with various distances, and then fit the impedance-distance value pairs with the well-known Least Square Estimation Algorithm
        into a polynomial function, with which we can calculate distance from a impedance data.
        \item[-] Also note this system is implemented with a 2-axis moving platform powered by G-code thus it is able to scan a surface automatically.
    \end{itemize}
\end{minipage}\\
\end{tabularx}

\begin{tabularx}{\linewidth}{ @{}X@{} }
    \href{https://github.com/leonezz/OpenChart.git}{\textbf{OpenChart}} as \textit{Creator and maintainer}.
    \signed{Jun. 2019 - Dec. 2019} \\[3.75pt]
    \begin{minipage}[t]{\linewidth}
        \begin{itemize}[nosep,after=\strut, leftmargin=1em, itemsep=3pt]
            \item[-] This is a personal project for charts making from data cheats, the aim of this project is to learn the programing pattern of Qt.
            \item[-] In this project, I designed a Web-based chart visualization page for charts rendering (ECharts is used to make charts),
            and a QWebEngine is utilized to load the page and execute the scripts.
            \item[-] For data transferring between the Qt/C++ end and the HTML/javascript end, I setted a socket server with QWebSocketServer
            at the Qt/C++ end and built a brige with QWebChannel to establish a connection.
            \item[-] During this project, I learned the usages of basic Qt widgets and the powerfull C++ libraries provided by Qt,
            as well as style customization techniques including CSS.
            \item[-] I also learned the multi-thread programing patterns during this project. To be specific,
            I utilize multi-threading to load data files in another thread such that the UI thread dose not stuck.
        \end{itemize}
        \end{minipage}
\end{tabularx}

\begin{tabularx}{\linewidth}{ @{}X@{} }
    \textbf{An embedded system for smart trash bin} as \textit{Creator and maintainer}.
    \signed{Dec. 2020 - May 2021} \\[3.75pt]
    \begin{minipage}[t]{\linewidth}
        \begin{itemize}[nosep,after=\strut, leftmargin=1em, itemsep=3pt]
            \item[-] In this project, I designed both the hardware and the software of an embedded system for a smart trash bin.
            \item[-] For the hardware part, I designed a core board for STM32F103RET6 MCU and a driver board for over 6 sensors
            and 1 executor as well as the MODBUS interface for data transferring. To summarize, 5 U(S)ART and 1 ADC of the MCU are utilized.
            \item[-] For the software part, I adapted a MODBUS slave protocol stack implementation to the MCU platform as the communication
            interface for interaction between MCUs and between MCU and PC, and I implemented all the drivers for the sensors and executor
            based on the STM32 HAL code base.
        \end{itemize}
        \end{minipage}
\end{tabularx}

%----------------------------------------------------------------------------------------
%	PUBLICATIONS
%----------------------------------------------------------------------------------------
\section{Publications}

\begin{tabularx}{\linewidth}{ @{}X@{} }
\textbf{FETL: Fusing Layer Representations for More Efficient Transfer Learning in NLP}. \signed{\textit{zhuwenq}. under review.}\\[3.75pt]
\begin{minipage}[t]{\linewidth}
    \begin{itemize}[nosep,after=\strut, leftmargin=1em, itemsep=3pt]
        \item[-] Following the nascent research area of Parameter Efficient Transfer Learning (PETL),
        we designed a simple yet effective architecture for PETL in NLP.
        \item[-] Inspired by the concepts of that knowledge about diverse aspects of language is embedded
        in different layers of PLM, and that not all of the knowledge is needed for a specific task,
        we propose to combine the representations from all PLM layers explicitly and individually
        with a attention-based fusion layer for a better transferring.
        \item[-] Our network works outside the PLM as a side-network, thus the backpropagation
        path does not involve the PLM with our method. Hence the training speed and training GPU
        memory usage are significantly better with our method than with other PETL methods.
        \item[-] Experiments on both low-resource and high-resource scenarios of the GLUE benchmark
        show that our method achieves promising results with negligible costs.
        \item[-] We also try to interpret the inner effects of the proposed fusion layer by analyzing
        the layer attention vectors task-wisely and token-wisely.
    \end{itemize}
    \end{minipage}
\end{tabularx}

%----------------------------------------------------------------------------------------
%	PUBLICATIONS
%----------------------------------------------------------------------------------------
%\section{Publications}
%\begin{refsection}[citations.bib]
%\nocite{*}
%\printbibliography[heading=none]
%\end{refsection}
\section{Internship}
\begin{tabularx}{\linewidth}{ @{}X@{} }
    \textbf{Nanjing Xunxing Electronic Technology Co., Ltd.}
    \signed{Jun. 2020 - Aug. 2020} \\[3.75pt]
    \begin{minipage}[t]{\linewidth}
        \begin{itemize}[nosep,after=\strut, leftmargin=1em, itemsep=3pt]
            \item[-] Participate in the development of embedded Linux software for the company's physical experiment equipment (a thermodynamic parameter tester), and I analyze software requirements with my physical knowledge.
            \item[-] I developed the driver and interactive software for the embedded equipment with the Qt framework, in which I implemented the functions of temperature sensor data reading and the heater controling through serial ports. A simple data analysis function based on the least square estimation algorithm is also implemented.
            \item[-] The software is finally deployed to the ARM device running the Ubuntu operating system. Since the device hardware has already been assembled, data can only be transferred to it through a USB drive. I wrote a deployment tool for code compilation and setting the software as the startup item with shell script.
        \end{itemize}
        \end{minipage}
\end{tabularx}
%----------------------------------------------------------------------------------------
%	SKILLS
%----------------------------------------------------------------------------------------
\section{Skills}
\begin{tabularx}{\linewidth}{@{}l}
\begin{minipage}[t]{\linewidth}
\begin{itemize}[nosep,after=\strut, leftmargin=1em, itemsep=3pt]
    \item[-] Modern C++, Python, Qt, and embedded systems.
    \item[-] Deep learning and NLP, transformer models especially.
    \item[-] Theoretical basis of computer science, including operating system, networking, the underlying principles of computers and a good foundation of algorithms and data structures.
    \item[-] Possess good documentation and engineering practice habits, and are willing to cooperate, communicate and discuss with others.
\end{itemize}
\end{minipage}
\end{tabularx}

\vfill
\center{\footnotesize Last updated: \today}

\end{document}
