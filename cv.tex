---
# Language (Optional)
lang: en
# Site Keywords & Description
keywords: Resume, zhuwenq,
description: This is zhuwenq's personal resume.
# Resume Title
resume_title: zhuwenq's Resume
# Job Applicant Name
name: zhuwenq
avatar: https://raw.githubusercontent.com/Leonezz/privateHosting/master/Leonezz.github.io.png
# Contact
contact:
  # Email
  - icon: fas fa-envelope
    text: Email
    url: mailto:zhuwenqa@outlook.com
  # Telegram
  - icon: fas fa-paper-plane
    text: Telegram
    url: https://t.me/ppingzi
  # Github
  - icon: fab fa-github
    text: Github
    url: https://github.com/leonezz
  # Blog
  - icon: fad fa-blog
    text: Blog
    url: https://zhuwenq-blog.netlify.app
  # URL
  - icon: fas fa-globe-europe
    text: https://zhuwenq-resume.netlify.app
    url: https://zhuwenq-resume.netlify.app
  # Phone Number
#  - icon: fas fa-phone-alt
#    text: 
#    url: tel:10086
# PDF Download Link
download:
  - title: Full
    icon: fas fa-download fa-fw
    url: https://github.com/leonezz

  - title: Simple
    icon: fas fa-download fa-fw
    url: https://github.com/leonezz
---


<center>
<a href='/'>English</a> | <a href='/zh-cn/'>简体中文</a>
</center>


## Intro

I am currently a postgraduate major in computer science, my research interests include Natural Language Processing, Parameter Efficient Transfer Learning and Information Extraction. Before my postgraduate's study, I graduate from the optical major of physics faculty.
In addition to my academic study, I am a person of wide interests, my interests include modern C++, embedded systems, advanced deep learning technics, reading books, photography and making coffee.
Contact me through:
- Email: [zhuwenqa@outlook.com](mailto:zhuwenqa@outlook.com)
- Telegram: [ppingzi](https://t.me/ppingzi)
- Github: [leonezz](https://github.com/leonezz)

## Education experience

- **Bachelor's degree** at **NUAA** major in *Optoelectronic Information Science and Engineering*, 2017 - 2021.
- **Master's degree** at **NUAA** major in *Computer Science and Technology*, 2021 - present.

## Project experience

### A Measuring System of Metal Surface Flatness Based on Electromagnetic Induction Principle
#### 2018.9 ~ 2018.11: Creator and maintainer.

- This project is supported by the NUAA Innovation and Entrepreneurship Program.
- We designed a tiny PCB coil in this project as the electromagnetic sensor and power this sensor with a dedicated chip.
- The dedicated chip collect the electromagnetic impedance value between the coil sensor and the surface of the object to be tested, and encode the value into digital data.
- We collect the data that represents the impedance (i.e. the distance between sensor and the surface to be tested) with various distances, and then fit the impedance-distance value pairs with the well-known Least Square Estimation Algorithm into a polynomial function, with which we can calculate distance from a impedance data.
- Also note this system is implemented with a 2-axis moving platform powered by G-code thus it is able to scan a surface automatically.

### [OpenChart](https://github.com/leonezz/OpenChart.git)
#### 2019.6 ~ 2019.12: Creator and maintainer.

- This is a personal project for charts making from data cheats, the aim of this project is to learn the programing pattern of Qt.
- In this project, I designed a Web-based chart visualization page for charts rendering (ECharts is used to make charts), and a QWebEngine is utilized to load the page and execute the scripts.
- For data transferring between the Qt/C++ end and the HTML/javascript end, I setted a socket server with QWebSocketServer at the Qt/C++ end and built a brige with QWebChannel to establish a connection.
- During this project, I learned the usages of basic Qt widgets and the powerfull C++ libraries provided by Qt, as well as style customization techniques including CSS.
- I also learned the multi-thread programing patterns during this project. To be specific, I utilize multi-threading to load data files in another thread such that the UI thread dose not stuck.

### A smart trash bin based on embedded systems
#### 2020.12 ~ 2021-5: Creator and maintainer.

- In this project, I designed both the hardware and the software of an embedded system for a smart trash bin.
- For the hardware part, I designed a core board for STM32F103RET6 MCU and a driver board for over 6 sensors and 1 executor as well as the MODBUS interface for data transferring. To summarize, 5 U(S)ART and 1 ADC of the MCU are utilized.
- For the software part, I adapted a MODBUS slave protocol stack implementation to the MCU platform as the communication interface for interaction between MCUs and between MCU and PC, and I implemented all the drivers for the sensors and executor based on the STM32 HAL code base.

## Publication
### FETL: Fusing Layer Representations for More Efficient Transfer Learning in NLP. *zhuwenq*. under review.

- Following the nascent research area of Parameter Efficient Transfer Learning (PETL), we designed a simple yet effective architecture for PETL in NLP.
- Inspired by the concepts of that knowledge about diverse aspects of language is embedded in different layers of PLM, and that not all of the knowledge is needed for a specific task, we propose to combine the representations from all PLM layers explicitly and individually with a attention-based fusion layer for a better transferring.
- Our network works outside the PLM as a side-network, thus the backpropagation path does not involve the PLM with our method. Hence the training speed and training GPU memory usage are significantly better with our method than with other PETL methods.
- Experiments on both low-resource and high-resource scenarios of the GLUE benchmark show that our method achieves promising results with negligible costs.
- We also try to interpret the inner effects of the proposed fusion layer by analyzing the layer attention vectors task-wisely and token-wisely.

## Skills

- C++ programing language, Qt, and modern C++.
- Deep learning and NLP, transformer models especially.
- Python programing language and pytorch, tensorflow, jax.
